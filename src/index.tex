%-------------------------------------------------------------------------------
%	CV/RESUME CONTENT
%-------------------------------------------------------------------------------

\cvsection{Professional Experience}

% {environment} {Job Title} {Organisation} {Location} {Dates}
\begin{cventry}{Data Consultant} {ASG Group}{}{2021 - Present}
  Malcolm is part of the Data Practice with ASG,
  where he is contributes mentoring and technical expertise.
  He is also a key member of the ASG Labs project,
  bringing data to life through
  proof of concept technology demonstrations.

  Malcolm has spent most of his time at ASG engaged with SA Power Networks
  building out their data analytics pipeline
  as part of standing up their data lake.
  Malcolm has led a technical team as part of an Agile project
  building out reports for regulatory submissions,
  managing and prioritising the
  tasks and workflows of all team members.
  Malcolm worked at all levels of the Azure data analytics stack
  including Azure Data Factory, Azure Data Lake, Synapse, Azure SQL, and Databricks
  architecting the technologies appropriate for each report.
  As part of standing up this environment,
  Malcolm has been developing best practices
  for data management, design patterns, and support,
  and leading the way in the use of the available tools.
  A core component of this best practice development
  is building out Machine Learning Workflows within Databricks,
  ensuring data provenance and the reproducibility of the models
  being developed and deployed.
\end{cventry}

% {environment} {Job Title} {Organisation} {Location} {Dates}
\begin{cventry}{Software Engineer (Machine Learning)}{Adelaide MRI}{}{2020 - 2021}
  Malcolm was part of a small team developing an internal web applicaiton
  used by Doctors to write medical imaging reports.
  Malcolm worked to integrate Machine Learning techniques 
  into this reporting application,
  improving the user experience when writing these reports
  by bringing sentences more likely to be used to the forefront
  reducing the need for time consuming searches.
  \begin{itemize}
    \item Developed Machine Learning pipelines to provide predictive
      suggestions during the report production process, reducing the time spent
      writing reports by 21\%.
    \item Optimised common database queries, reducing the execution time by
      up to 90\%.
    \item Responsible for architecting solutions
      within the python back-end, including;
      designing database models in SQLAlchemy,
      designing RESTful APIs,
      optimising API performance,
      and incorporating security updates into the development process.
    \item Mentored graduate students in applying programming and data science
      best practices.
  \end{itemize}
\end{cventry}

% {environment} {Job Title} {Organisation} {Location} {Dates}
\begin{cventry}{eResearch Trainer, eResearch Training Administrator}{Intersect Australia}{}{2018 - 2020}
  Malcolm worked at Intersect to develop and deliver training 
  for a range of programming tools
  targeted to researchers at Universities across Australia.
  Malcolm worked at all stages of the training process, 
  includign the design of content, 
  the preparation of material, managing attendees, 
  content delivery, and review of course performance.
  \begin{itemize}
    \item Developed and documented best practices for delivering Online
      training for programming languages during COVID-19 resulting in an
      18 point Net Promotor Score improvement over 
      the same suite of courses delivered in person.
    \item Developed and presented a webinar
      \chref{https://youtu.be/ViUIXTIVCvs}
      {"A Showcase of Data Analysis in Python and R: A case study using COVID-19 data"}
      demonstrating best practices in data analysis on a dataset.
    \item Delivered online training courses across a range of eResearch topics
      including Python, R, Julia, Git, SQL, MATLAB, and HPC.
    \item 100\% Net Promoter Score when runnning Programming using Julia and Programming using R.
  \end{itemize}
\end{cventry}

% {environment} {Job Title} {Organisation} {Location} {Dates}
\begin{cventry}{Co-Founder, Chief Scientist}{FluroSat}{}{2016 - 2017}
  Malcolm Co-Founded the startup FluroSat, 
  implementing remote sensing and image processing techniques 
  from academic research to target key decision points 
  within the agricultural value chain.
  \begin{itemize}
    \item Implemented scalable workflows for processing remote sensing data on AWS.
    \item Developed tools for image processing and analysis in Python.
    \end{itemize}
  \textbf{Achievements}
  \begin{itemize}
      \item Portfolio company in the Muru-D accelerator program
      \item Best Team, Cicada Innovations Nobel Gala Dinner
      \item Finalist, World Bank Big Data Challenge
  \end{itemize}
\end{cventry}

\cvsection{Education}


\begin{cventry}
  {Doctor of Philosophy (PhD)} % job title
  {The University of Sydney} % organization
  {} % location
  {2016 - 2020} % date(s)
  Malcolm used a combination of statistical mechanics and data science to
  study the translational and rotational dynamics
  of small molecules in the liquid phase.
  He used High Performace Computing
  including MPI and GPU Compute
  to run long timescale computer simluations,
  generating over 10 TB data for anaylsis.
  Malcolm built a data lake and data processing workflow
  to handle the complex analysis of this large dataset,
  allowing rapid iteration and testing of ideas.
  \begin{itemize}
    \item Created interactive dashboard in python for analysis of simulations
      combining Bokeh and scikit-learn for real-time machine-learning analysis.
    \item Used infrastructure as code to define reproducible simulation environments
      using Ansible and Docker.
    \item Developed optimised data processing tools in Python, Cython, C++, and Rust,
      using software developent best practices to ensure high quality code.
    \item Designed effective data visualisations for complex datasets.
  \end{itemize}
  \textbf{Achievements}
  \begin{itemize}
    \item Winning team, Inventing the Future Competition 2016.
    \item Finalist, Merck Innovation Challenge 2018.
    \item Journal of Materials Chemistry C Poster Prize 2018.
  \end{itemize}
\end{cventry}

\begin{cventry}
  {Bachelor of Science (Advanced) Honours} % job title
  {The University of Sydney} % organization
  {} % location
  {2010 - 2015} % date(s)
  In his Honours thesis,
  Malcolm used molecular dynamics simulations of small rigid molecules
  to investigate the effect of shape on structural and dynamic properties.

  \textbf{Achievements}
  \begin{itemize}
    \item Completed with \nth{1} Class Honours and \nth{1} in his cohort.
    \item Received the Walter J Moore Honours Scholarship for academic merit in Honours.
    \item Sydney Abroad International Exchange Scholarship.
    \item Dean of Science Undergraduate Exchange Scholarship.
    \item Received Levy Scholarship for achievement in \nth{1} year chemistry.
    \item Undertook research through the Summer Scholarship Program.
  \end{itemize}
\end{cventry}

\cvsection{Software Skills}

\begin{cvskills}

\cvskill
  {Data Science} % Category
  {Python, Numpy, Pandas, scikit-learn, Keras, xgboost, R, Julia, SQL, matplotlib}

\cvskill
  {Data Engineering} % Category
  {SQL, Python, Azure Data Factory, Azure Synapse, Databricks}

% \cvskill
%   {Backend} % Category
%   {Python, SQLAlchemy, flask, marshmallow}

\cvskill
  {Frontend} % Category
  {Javascript, Web Assembly (WASM), CSS, HTML}

\cvskill
  {Cloud} % Category
  {Docker, AWS, Azure}

\cvskill
  {Systems} % Category
  {Rust, C++, C, CUDA, MPI}

\cvskill
  {Other} % Category
  {Git, \LaTeX, Linux, MacOS}

\end{cvskills}
