\documentclass[12pt,a4paper]{moderncv}

\makeatletter
\def\input@path{{classes/}}
\makeatother

\usepackage[utf8]{inputenc}
\usepackage{publications}
\usepackage[super]{nth}
\usepackage[scale=0.85]{geometry}
\usepackage{tikz}

\moderncvstyle{casual}
\moderncvcolor{blue}

% Change the width of the column with dates
\setlength{\hintscolumnwidth}{2.5cm}

\addbibresource{bibliography/publications.bib}

\name{Malcolm}{Ramsay}
\title{Computational Chemist}
\phone[mobile]{0466~224~898}
\email{malramsay64@gmail.com}
\homepage{github.com/malramsay64}
\homepage{malramsay.com}
\social[linkedin]{malramsay64}
\social[twitter]{malramsay64}
\social[github]{malramsay64}


\newcommand{\cvreference}[5]{%
    \textbf{#1}\newline%
    \ifthenelse{\equal{#2}{}}{}{#2\newline}%
    \ifthenelse{\equal{#3}{}}{}{#3\newline}%
    \ifthenelse{\equal{#4}{}}{}{\phonesymbol~#4\newline}%
    \ifthenelse{\equal{#5}{}}{}{\emailsymbol~\href{mailto:#5}{#5}\newline}%
}

\newcommand{\daterange}[2]{%
  % This ensures the correct spacing
  \vspace{-2em}
  \begin{center}
    #2 \\
     | \\
    #1
  \end{center}
  \vspace{-2em}
}

\newcommand{\chref}[2]{\href{#1}{\color{color1}#2}}

\newcommand\cvskills[4]{%
  \cvcomputer{}{\skills{#1}{#2}}{}{\skills{#3}{#4}}
}


\newcommand\skills[2]{%
  #1 \hfill%
  \begin{tikzpicture}
    \draw[fill=color2,color2, opacity=0.3] (0, 0) rectangle (3, 0.3);
    \draw[fill=color1,color1] (0, 0) rectangle (#2 * 3, 0.3);
  \end{tikzpicture}
}



%\recipient{<Company>}{<Address>}
\recipient{~}{~}
\date{\today}
\opening{Dear Sir/Madam,}
\closing{Regards,}

\begin{document}

\makelettertitle

I am writing to apply for the position of C++ Software Engineer - Communications, Media Server Engineering. 

% B.S. or M.S. in Electrical Engineering, Computer Science, Computer/Software Engineering (HD or D average preferred).

I have just completed a Bachelor of Science (Advanced) with 1st Class Honours and placing 1st in my cohort. I completed majors in both Computational Science and Chemistry with a Distinction average throughout my degree.

During my Honours project I had to develop a program to extract useful data from my simulated system. For this I wrote a program in C++ (\href{http://github.com/malramsay64/analysis}{github.com/malramsay64/analysis}), constantly developing and expanding on the program to add new functionality as I learnt more about the features of the model systems I was studying. Using this program I was able to efficiently extract data from a large number of model systems which enabled me to make important discoveries which are in the process of being published.

% A Quick learner

During my Honours project I was was required to use a wide array of computational tools and programs that I had previously not had experience using. During the course of my Honours project I very quickly became proficient using these tools, heavily incorporating C++, Git, GNU Make and Gnuplot into the code base of my project while previously not having used any of them.

% Desirable Qualities:

% Experience developing software on multiple platforms

During my honours project I was developing software for a number of different systems. My main computer, on which I did the majority of my development was a Mac. Along with this computer I had to develop programs to run on a computing cluster running a Linux variant, a secondary workstation running Ubuntu, and a Raspberry Pi running Debian as a backup. This required testing code on each system and ensuring that each different system could be identified to deal with the idiosyncrasies of each system. 

% Experience developing multithreaded applications

I have had experience developing multithreaded programs using a variety of different approaches. During my undergraduate degree I took courses in parallel programming which introduced me to parallel programming using pthreads, MPI, cilk, and CUDA. I attempted to use some of these techniques to speed up my honours project, however after testing parallel versions of the code, I found that running multiple analyses concurrently using the parallelisation available in GNUMake was faster.

% Experience developing efficient software

In the development of the code for the analysis in my honours project I had to process many large input files each on the order of a gigabyte. This meant it was not feasible to hold every time point in the file in memory for the duration of the analysis. Instead to get the nonlinear analysis that I required, I stored a subset of all time points as key frames to reference back to while only storing the remaining time points for the duration of the analysis. This significantly reduced the memory requirements allowing multiple instances to run concurrently with no memory issues.

% Knowledge of cryptography and security

With the number of high profile data breaches I have a reasonably good understanding of computer security and cryptography. Passwords should always be stored encrypted with a salt, ideally using a hash harder to compute than MD5. Also important is the protecting of any html form input, replacing special characters to prevent access to the back-end database or allowing javascript to run. In terms of practical applications my personal computer is accessible remotely over the internet. However it is only accessible using ssh on a non-standard port and authenticating using a public/private key pair.

\makeletterclosing

\end{document}
